
\section{Introduction}

The question of how family size affects outcomes for children has received considerable attention in the literature on fertility. The quantity-quality model, introduced by \textcite{Becker1960} and formalized in \textcite{Becker1973} and \textcite{Becker1976}, predicts that there is a trade-off between the number (quantity) of children and their “quality”. Using a utility maximization framework, the model shows that the cost or shadow price of quality increases as the number of children increase, and vice versa; this generates a trade-off between quality and quantity. One implication of this is that an increase in the number of children, keeping other factors fixed, will lower investment per child and result in children of lower quality.

Early empirical evidence on the quantity-quality relationship relied on showing a negative association between family size and outcomes for children (usually education).  However, taking these results to be evidence of a causal relationship is challenging because family size is endogenous; the number of children that parents desire to have is related to both observed and unobserved characteristics of the parents that also affect child outcomes \parencite{Black2010}. For instance, childbearing decisions are affected by the education of the parents, their earning potential, career ambition, lifestyle, and expectations about the benefits and costs of having a child \parencite{Angrist2006,oberg_casual_2021}. These are factors that also affect how much parents invest in the human capital of their children. Since most of these factors are unobserved or even unobservable, naïve attempts to estimate effect, such as ordinary least squares (OLS) regression, are likely to suffer from omitted variable bias \parencite{oberg_casual_2021}.

Studies that tried to uncover the causal effect of fertility on individual outcomes have therefore tried to exploit “natural experiments” that lead to an exogeneous increase in family size. The two well known examples of such natural experiments used as the basis for instrumental variables estimation in the literature are the birth of twins and the sex composition of older siblings (given a preference for mixed-sex siblings, parents are more likely to have another child if the sex of the first two/three children is the same). However, the majority of previous research that use either or both of these instruments were conducted in a high-income or middle-income country setting. Despite the fact that family planning policies are a more important issue in poorer countries, there is a paucity of relevant empirical evidence coming from these countries. In particular, there are very few studies on this topic from Africa, largely due to the lack of data. 

This study attempts to fill this gap by providing empirical evidence on the effect of sibling size on the educational attainment of children using census data from South Africa. Following the literature, I used both the occurrence of twin births and the sex composition of children in earlier births to instrument for family size. For the twins instrument, I considered parity specific twin births (twins at second birth and twins at third birth), and looked at outcomes for non-twin older children. This gave rise to two samples of school aged children: firstborns in families with at least two kids (called the 2+ sample) and first and second-born siblings in families with at least three children (called the 3+ sample). Similarly, the sex composition instrument was taken to be an indicator for the matching of the sexes of the first two children (for the 2+ sample) and the first three children (for the 3+ sample). First stage results show that both the occurrence of twin births and the sameness of sex in earlier births significantly increase the number of children. This holds for both the 2+ and the 3+ samples and in all the subsamples I considered.

The use of multiple instruments for the number of children is not common in the literature. Given the fact that the results from the two types of instruments have distinct features, this approach offers an advantage over studies that use a single instrument. Two important differences about the effects captured by the two instruments are worth pointing out. First, each instrument captures an effect for different segments of the population. This follows from the fact that any estimate that employs IV strategy captures the effect on individuals affected by that instrument \parencite{imbens_identification_1994}. These individuals are called the \enquote{compliers} \parencite{angrist_identification_1996}. It turns out that the group of compliers for twins and sex composition instruments are different. As will be argued later, twins instruments identify the effect of treatment on the whole population of the nontreated, whereas the compliant population for the same sex instrument is less than complete \parencite{Angrist2006,Angrist2009}. Second, as will be explained below, each instrument captures a causal effect over different ranges of sibling size. In particular, the twins instrument measures the effect of fertility at parities closer to the parity of the occurrence of twins. The same sex instrument, on the other hand, captures an effect over wider parities. 

Moreover, the two instruments are potentially subject to different types of omitted variable biases. For example, the likelihood of occurrence of twins vary with the characteristics of the mother such as age at birth, ethnicity, and health seeking behaviour. But such concerns do not arise with the same sex instrument. Instead, a challenge to the exclusion restriction with respect to the same sex IV is the possibility of gender-specific economies of scale that could reinforce child quality investments when parents have children of the same sex (Rosenzweig and Wolpin, 2000). This fact highlights a further potential advantage of using multiple instruments for family size: given the different types of endogeneity threats that the two instruments face, a comparison of the estimates using each IV can serve as a specification check \parencite{angrist_multiple_2010}.

Because of limitations on the data, the study focuses on immediate short run outcomes that relate to the schooling of children. In particular, it focuses on the grade achievement of children by the time of the census. Two variables that measure educational attainment were constructed taking into account institutional factors that could cause a variation in the grade level achieved by a child at a particular age. The first is an educational attainment index, a continuous variable that measures whether each child is at, above, or below the “appropriate” grade, and the second is an indicator for whether the child is lagging behind in grade relative to his/her peers. Moreover, two more intermediate variables were considered: whether the child attends a private school and the labour force participation status of the mother. These can be considered as inputs into child quality that reflect investment in children’s human capital \parencite{caceres-delpiano_impacts_2006}.

Results from OLS regression point to a negative correlation between family size and children’s educational attainment. However, 2SLS regression estimates using both twins and sex composition instruments show that sibling size has no effect on the educational attainment of children and other intermediate outcomes. Heterogeneity analysis conducted using generalized random forests \parencite{Athey2019} indicates this result holds across different subsamples with different sociodemographic characteristics. These findings are consistent with results from previous studies that find a negative association but no negative effect when using one or both of these instruments \parencite[see, for example,][]{Black2005,Black2010,caceres-delpiano_impacts_2006,angrist_multiple_2010,bhalotra_twin_2020}.\footnote{See also the reviews in \textcite{clarke_children_2018} and \textcite{oberg_casual_2021}.} Hence, this paper is the latest in a growing body of empirical literature that challenge the existence of a quantity-quality trade-off.

The paper proceeds as follows. Section II describes the data and samples used, the outcome variables, and the empirical framework for the IV strategy. Then, Section III discusses the results, addresses challenges to validity, and presents findings from heterogeneity analysis that were conducted using generalized random forests. Finally, Section IV concludes.
