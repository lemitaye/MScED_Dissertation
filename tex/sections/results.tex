
\section{Results and Discussions}

In this section, I present and discuss the main regression results on the effect of the number of children on different outcomes. I will first present and discuss the main regression estimates found using the whole population of firstborns (2+ sample) and the population of first and second borns (3+ sample), which are reported in \autoref{tab:03}. The subsections that follow discuss possible challenges to the validity of the results and a subsample analysis.

\autoref{tab:03} presents OLS and 2SLS estimates of the effect of the number of children on the four outcome variables discussed in section ?. OLS results, reported in column 1 (for the 2+ sample) and column 5 (for the 3+ sample), point to the expected negative association between family size and favourable outcomes. We have the usual story that kids who grow up in larger families have lower educational attainment, are more likely to be left behind in school, and are less likely to attend private school. Mother’s are also less likely to participate in the labour market when they have many children. It is also important to point out that these estimates are very precise: their standard errors range from .001 to .003. However, these results are only indicative and, in the presence of possible endogeneity of family size, cannot be taken as showing causality.

In contrast to the OLS results, 2SLS estimates consistently point to a statistically insignificant or even positive effects of family size. The 2SLS results using Twins instruments are shown in column 2 (for the 2+ sample) and column 3 (for the 3+ sample) of \autoref{tab:03}. All of the estimates are statistically insignificant at conventional levels and even some have unexpected signs. For example, the effect on the educational attainment index in the 2+ sample using Twins2 as an IV is .005 (se = .006) while the corresponding effect in the 3+ sample (using Twins3 as an IV) is .012 (se = .009). Similarly, the 2SLS estimates of the impact on the dummy variable “Left Behind” are statistically insignificant and have the unexpected signs in both the 2+ and 3+ samples. The lack of statistical significance using twins instruments also extend to the variables that relate to child investment: private school attendance and the mother’s labour force participation. 

Similar results hold when using sex composition instruments. 2SLS estimates using both Boy12 and Girl12 instruments show no effect of the number of children on any of the outcomes. Similar results hold in the 3+ sample when using Boy123 and Girl123 instruments together. The only exception is the effect on educational attainment [.052 (se = .028)], which surprisingly, has an unexpected sign and is statistically significant only at the 10\% level. The results for the probability of being left behind in the 3+ sample and mother’s labour force participation in both samples, although insignificant, also have the opposite signs to the OLS estimates.

Columns 4 and 8 of \autoref{tab:03} report 2SLS results using an instrument list that combines the twins and sex composition instruments in each sample. The purpose of combining these two different sets of instruments is to increase precision and thereby produce a single more efficient IV estimate (Angrist et al., 2010). Indeed, the standard errors of the estimates in column 4 (2+ sample) and column 8 (3+ sample) are lower than the corresponding estimates using each group of instruments separately (columns 2 and 3 for the 2+ sample; and columns 6 and 7 for the 3+ sample). The estimates obtained using the combined instruments bolsters the null results that is obtained using each instrument set separately: none of the estimates obtained are statistically significant at the 5\% level and only two results are significant at the 10\% level. The estimate for the effect on educational attainment in the 3+ sample [.016 (se = .008)], although significant at the 10\% level, has an unexpected sign. The effect on private school attendance in the 2+ sample, which is -.017 (se = .010), is the only other estimate that is significant at the 10\% level.

As discussed in the previous section, the two sets of instruments capture an average causal response for different subpopulation of compliers. We have also shown that each instrument shifts the fertility distribution very differently. Now, given the similar set of 2SLS estimates they generate, one wonders if these results are pointing to a homogenous effect, i.e., a common family size effect of zero for just about everybody. I conducted an overidentification test to formally test the equality of the 2SLS estimates generated by the two sets of instruments. The test results (not presented, but available upon request) show that we cannot reject the null of equality of the estimates for the two instrument sets across all of the outcomes in both samples. In this setting, the traditional overidentification test becomes a formal test for treatment effect homogeneity.  But, as Angrist and Piscke (2009) point out, this requires maintaining the hypothesis that all instruments in the overidentified model are valid, a topic we turn to next.
