
\section{Data and Methods}

The data comes from the 10\% public use sample of the 2011 South African Census. The micro data at the individual level has observations for each person within a household from all nine provinces of South Africa. I retained only sons or daughters in each household whose biological parents were alive by the time of the census and matched them with data on their mothers. After identifying the order of birth among two or more siblings using their dates of birth, I selected those observations where the age of the firstborn child is less than or equal to 18. Finally, two groups of samples were used in the final analysis. The next subsection discusses these in detail.

\subsection{The 2+ and 3+ samples}

The 2+ sample consists of only firstborn children with one or more siblings (so the total number of children per mother would be two or more). I selected only firstborns aged 8 to 18.\footnote{ South Africa has a compulsory schooling law where children who have turned 6 by June 30 are required to attend grade 1 (see \url{https://www.education.gov.za/Informationfor/ParentsandGuardians/SchoolAdmissions.aspx}). To look at 8 year olds and above would allow me to observe variations in educational achievement and grade retention. }  The twins IV (Twins2) in this sample is an indicator for whether the second birth was a multiple birth or not. But I have excluded firstborns who are twins themselves because of the special characteristics of twins that could confound results. Hence, all firstborns in this sample are singletons. Similarly, the same sex IV (SameSex12) is an indicator for whether the sex of the first and second births are the same. In order to look at the effect by gender, I disaggregated this into two separate indicators: whether the first two births were both boys (Boy12) or both girls (Girl12).\footnote{ The sum of Boy12 and Girl12 gives the SameSex12 dummy, i.e., SameSex12 = Boy12 + Girl12. }  

The first two columns of \autoref{tab:01} present summary statistics on the variables in the 2+ sample. The sample consists of almost 95,000 firstborns and has a balanced sex composition. The average number of kids is 2.7 and the mean age of the firstborns is 13. About three-quarters of the observations belong to Black African mothers, while coloured and white mothers each account for approximately 11\%. The rate of occurrence of twins in the second birth is 1\%. On the other hand, about 52\% had a sibling of the same gender from second birth. Given that 50\% of the sample are males, we have an almost equal proportion of matching first two births for each gender: 26.1\% for boys and 25.5\% for girls. 

The 3+ sample is a subsample of the 2+ sample, where both non-twin first and second-born kids with one or more siblings are included. Here also, I selected children aged 8 to 18 since I can only observe variations in schooling outcomes for kids aged 8 or older. The twins IV in this sample (Twins3) is a dummy for whether the third birth was a twin birth or not. The same sex IV, on the other hand, is an indicator for whether the first three kids were of the same sex. This is termed SameSex123. As in the 2+ sample, I have disaggregated this indicator into two separate indicators based on the sex of children in the first three births. The Boy123 and Girl123 are indicators for whether the first three children are all boys or all girls, respectively. 

Columns 3-6 of \autoref{tab:01} presents summary statistics on the 3+ sample. Columns 3 and 4 focus only on firstborns while columns 5 and 6 include both first and second-borns. There is no substantial difference across many of the variables between the firstborns in the 3+ sample and their counterparts in the 2+ sample. A similar result follows when looking at the entire 3+ sample (i.e., both first and second-borns; $ N = 57,358 $). The average family size in this sample is 3.6 and the average age of the children is 13. Just as in the 2+ sample, we have an almost perfectly balanced sex composition. The twin rate at the third birth, 1.3\%, is a little higher than in the 2+ sample. About 27\% of the observations are in households where the first three births are of the same sex. Again, this is well balanced across the two sexes (about 13.5\% each). Black African mothers now account for a larger proportion (82\%), while the share of white mothers declined to 6.1\%. One can also see that the share of mothers with no schooling is higher in the 3+ sample compared to the 2+ sample (8\% vs. 5\%).   
    

\subsection{Model Specification and Outcome Variables}

The structural equation that captures the causal relationship of interest is

\begin{equation}\label{eq:01}
	y_{i} = X_{i}\boldsymbol{\beta} + \gamma n_{i} + \epsilon_{i}
\end{equation}

where $ y_{i} $ is an outcome variable, $ X_{i} $ is a vector of covariates including characteristics of the children and their mothers, and $ n_{i} $ is the total number of children (including the subject) in the household. As discussed in the introduction, because the number of children, $ n_{i} $, is endogenous, OLS estimation of \eqref{eq:01} gives a biased and inconsistent estimate of $ \gamma $. I use both the birth of twins and the sameness of sex in earlier births to instrument for family size. The rest of the current subsection discusses the outcome variables explored in the study. The next subsection has a detailed discussion on the proposed instruments.  

The study will look at the effect of sibship size on four outcome variables: (i) educational attainment, (ii) whether a kid is lagging behind in grade, (iii) whether the kid attends a private school, and (iv) the labour force participation of the mother. The first two are final outcome variables whereas the second two are considered as intermediate outcome variables \parencite{caceres-delpiano_impacts_2006}. 

Following \textcite{rosenzweig_testing_1980}, an educational attainment index was constructed by taking the ratio of the highest grade achieved by a child at the time of the census to the mean grade of all kids of the same age, month of birth, and province. The reason month of birth was chosen as a grouping variable is that laws on admission are based on the age of a child at a specific point in time during the admission year.  Hence, the month of birth is one determining factor for the variation in grade achievement of children of the same age. However, individual provinces have their own rules on registration timelines as well as implementation. From \autoref{tab:01}, the educational attainment index has a mean of 1 and a standard deviation of about 0.16 in both the 2+ and 3+ samples. [Figure 1] shows a histogram of this index in both samples, which provides a more informative picture of its distribution. 

Similarly, a dummy for being left behind is constructed by comparing the highest grade of a child to the mode of the highest grade of all kids of the same age, month of birth, and province in the relevant sample. It takes a value of 1 if a child's highest grade is below the mode and 0 if it is equal to or above the mode. This essentially measures the same thing as the educational attainment index but has some advantages over the latter. Particularly, as it uses mode instead of mean, it is not affected by extreme values. But as a discrete measure it does not take into account the extent of the lagging (or progressing). It can be considered as a useful complement to the educational attainment index. According to \autoref{tab:01}, 38.6\% in the 2+ sample and 37.6\% in the 3+ sample are lagging behind using this measure. 

The third outcome variable, \enquote{private school} is a dummy for whether the child attends a private school ($ = 1 $) or a public school ($ = 0 $). Like in many countries, private schools offer better quality education and are preferred to public schools in South Africa. They are also more expensive. Thus enrollment in private school can reflect costly parental investment in the education of children. Attending a private school can also lead to better educational outcomes. The available evidence (?) suggests that educational outcomes are better for kids attending private school, although we should be careful to interpret this as showing causation, not merely correlation \parencite{caceres-delpiano_impacts_2006}. From \autoref{tab:01}, about 9\% in the 2+ sample and 6.3\% in the 3+ sample attend private school. These might sound low but are not surprising given that private schools are expensive in South Africa.

The last outcome variable, mothers' labour force participation (Mothers' LFP) is also a dummy for whether a mother is in the labour force ($ =1 $) or not ($ =0 $). The relationship between a mother's labour market outcome and family size has been explored in many studies (\textit{cite a few studies}). As an intermediate outcome, a mother being in the labour force has ambiguous effects on child outcomes \parencite{caceres-delpiano_impacts_2006}. On the one hand, working mothers generate income that they can invest on their children, but, on the other hand spend less time caring for their children. Either way, it is an important channel between fertility and the welfare of children. According to Table 1, about 76.3\% of mothers were in the labour force in the 2+ sample, while the corresponding figure in the 3+ sample is 71.3\%.


\subsection{Instrumental Variables, Interpretation, and First Stages}

The main motivation behind the use of instrumental variables is to uncover a causal relationship between family size and outcomes for children and mothers. Following the potential outcomes framework, what we are trying to estimate using the birth of twins and the sameness of sex of the first two births as natural experiments for the number of children is what \textcite{Angrist1995} called the Average Causal Response (ACR). \textcite{Angrist1995} have shown that, given mild regularity assumptions, the Two Stages Least Squares (TSLS) estimates using a binary instrument $ Z $ to instrument for an endogenous variable that takes a finite set of integer values  (say $ 0, 1, \dots, J $) is the weighted average of per-unit causal effects (that is, the effect of going from 0 to 1, the effect of going from 1 to 2, and so on) along the length of an appropriately defined causal response function. More formally, let $ N_{0i} $ and $ N_{1i} $ be the potential number of children in the family if a binary instrument $ Z_{i} $ were equal to 0 and 1, respectively. The observed number of children $ n_{i} $ is then, $ n_{i} = N_{0i} + (N_{1i} - N_{0i})\cdot Z_{i} $. Similarly let $ Y_{i}(j) $ be the potential outcome as a function of $ j $, where j takes possible values for $ n_{i} $ in the set $ \{0, 1, \dots, J\} $. Only one of these potential outcomes is realized and we denote the observed outcome by $ y_{i} $. In the simplest case with no covariates, the IV estimator using a binary instrument $ Z $ gives the Wald estimator.\footnote{The interpretation of the ACR is more elaborate when covariates are added, but the basic idea is preserved. See \textcite[p.~437]{Angrist1995}.} Then,


\begin{equation*}
	\beta_{w} = \dfrac{\mathbb{E}(y_{i} | Z_{i} = 1) - \mathbb{E}(y_{i} | Z_{i} = 0)}{\mathbb{E}(n_{i} | Z_{i} = 1) - \mathbb{E}(n_{i} | Z_{i} = 0)} = \sum_{j = 1}^{J} \omega_{j}\cdot \mathbb{E}(Y_{i}(j) - Y_{i}(j-1) | N_{1i} \geq j > N_{0i})
\end{equation*}

where
\begin{equation*}
\omega_{j} = \dfrac{\mathbb{P}(N_{1i} \geq j > N_{0i})}{\sum_{i = 1}^{J} \mathbb{P}(N_{1i} \geq i > N_{0i})}
\end{equation*}
\vskip10pt

The unit causal response, $ \mathbb{E}(Y_{i}(j) - Y_{i}(j-1) | N_{1i} \geq j > N_{0i}) $, is the average difference in potential outcomes for \textit{compliers} at point $ j $, that is, individuals driven by the instrument from a treatment intensity less than $ j $ to at least $ j $.\footnote{We are using the word \enquote{compliers} in the context of the treatment-effects framework outlined in \textcite{angrist_identification_1996}. See the discussion below for more explanation.} Thus, the Wald estimator, $ \beta_{w} $, is \enquote{a weighted ACR for people from families induced by an instrument to go from having fewer than j to at least j children, weighted over j by the probability of crossing this threshold.} \parencite[p.~787]{angrist_multiple_2010}. The numerator in the expression for the weights $ \omega_{j} $ above, $ \mathbb{P}(N_{1i} \geq j > N_{0i}) $, represents the proportion of compliers at point $ j $.\footnote{Individuals with $ N_{1i} \geq j > N_{0i} $  for any j in the support of $ n_{i} $ are considered as compliers.} This is normalized by $ \sum_{i = 1}^{J} \mathbb{P}(N_{1i} \geq i > N_{0i}) $, which can be shown to be equal to the Wald first stage \parencite[see][p.~183]{Angrist2009}. That is,

\begin{equation*}
\mathbb{E}(n_{i} | Z_{i} = 1) - \mathbb{E}(n_{i} | Z_{i} = 0) = \sum_{i = 1}^{J} \mathbb{P}(N_{1i} \geq i > N_{0i})
\end{equation*}

There are three assumptions that lie behind the ACR theorem: (i) independence and exclusion; (ii) the existence of a first Stage; and (iii) monotonicity \parencite{Angrist2009}. The independence assumption requires that the instrument be independent of potential outcomes and potential treatment assignments. This is sometimes alternatively stated by saying that the instrument need to be as good as randomly assigned. Both the twins and same sex instruments were originally proposed as natural experiments for fertility on the ground that both are virtually randomly assigned \parencite{rosenzweig_testing_1980,angrist_children_1998}. However, this is no longer obvious because various threats to the randomness of both instruments have been proposed in the literature (see the discussion below). The exclusion assumption, on the other hand, says that the instrument used has no effect on the outcomes other than through its effect on family size.  Although this is closely related to the independence assumption, it is distinct from it. The exclusion restriction is a \enquote{a claim about a unique channel for causal effects of the instrument} \parencite[p.~153]{Angrist2009}. Again, various authors have pointed out ways in which exclusion restriction might fail to hold for both instruments. I have tried to address some of the important concerns in the upcoming subsections. 

The other important assumption needed to estimate the ACR is monotonicity. To understand the idea behind this assumption, we need to introduce the four mutually exclusive groups comprising of our quasi-experimental population: compliers, always takers, never takers, and defiers.These were first outlined in the treatment-effects framework of \cite{angrist_identification_1996}.  Always takers always get treated regardless of their assignment, never takers never get treated regardless of their assignment, and compliers get treated when the instrument is switched on and don't get treated otherwise. The defiers, on the other hand, are a very strange group who get treated when the instrument is switched off and don't get treated when it is switched on. The assumption of monotonicity rules out the presence of defiers since they complicate the link between ACR and the reduced form \parencite{Angrist2009}.\footnote{\cite{Angrist2009} discuss monotonicity in the context of the LATE framework of \cite{imbens_identification_1994}. But this carries over directly to the ACR since the ACR is just an extension of the LATE \parencite[see][p.~181]{Angrist2009}.}  In other words, monotonicity rules out the case where the instrument pushes some people into treatment while pushing others out. In our case, monotonicity requires that the proposed instrument moves fertility in one direction only. In particular, we assume that the potential number of children when the instrument is switched on is at least as large as it would have been when it is switched off; i.e., $ N_{1i} \geq N_{0i} $. This is automatically satisfied by the twins instrument as fertility is always increased because of a twin birth for any mother. However, monotonicity need not hold for the same sex instrument as there could be parents who prefer children of the same sex and therefore go on to have another child if the sexes of the first two or three children are different \parencite{Huber2015}. In the subsection on the same sex instrument, I discuss a partial check for monotonicity using first stage regressions. 



\subsubsection{The Twins Instrument}

The first stage for the twins instrument is

\begin{equation}\label{eq:02}
	n_{i} = X_{i}\boldsymbol{\theta} + \delta\cdot {\rm Twins }_{i} + \xi_{i}
\end{equation}

where $ {\rm Twins }_{i} $ is an indicator for the occurrence of twins in the second (Twins2) or third (Twins3) birth, corresponding to the 2+ and 3+ samples, respectively. There are two rationales for considering twins at a fixed parity and looking at outcomes for older non-twin children. First, it accounts for the fact that twins are more likely in a larger family. Hence, using an indicator for a twin birth at any parity as an IV will be invalid \parencite{oberg_casual_2021}. Second, looking at older children before the twin birth avoids the confounding effect from birth order that would occur after a twin birth \parencite{Black2010}. For instance, if the second birth is a twin birth, then a younger sibling will be a fourth child. So, the twin birth affects both family size and the birth order for any siblings in later births. The obvious drawback of focusing on children prior to the occurrence of twins is that it doesn't allow to look at the effect of birth order on the children.  

Column 1 of \autoref{tab:02} report the first stage results for the twins instruments. In both the 2+ and 3+ samples, twins lead to a positive and statistically significant increase in the number of children. A twin second birth increases the number of children by 0.83 whereas a twin third birth leads to 0.76 additional children, controlling for the other covariates in the model. These estimates remain virtually the same after controlling for same sex dummies by gender (column 4).  The twins instrument is also relevant in both samples as evident from the sufficiently large F statistics from the Wald test for exclusion restrictions (963 for the 2+ sample; 484 for the 3+ sample). 

In addition to being relevant, the twins instrument needs to satisfy the exclusion restriction. A related assumption required by the LATE framework is that it also be random or at least as good as randomly assigned. The randomness of twin births has been challenged in a number of ways. There is a common finding that older mothers are more likely to give birth to twins than younger mothers \parencite{angrist_multiple_2010}. I also found this to be the case in the 2+ sample but not the 3+ sample. Mothers who give birth to twins in their second birth are about a year older at first birth compared to mothers who don’t. The age of the mother is also related to the validity of the instrument in another way. An important concern with twins is that they affect both the number of children and the timing of a third (fourth) child for families with two (three) children. The only families for whom twins increase family size are those who have an unintended extra child because of the birth of twins \parencite{oberg_casual_2021}. That is, they have one more child than the desired number of children. For parents who haven’t attained their desired number of children, a parity-specific twin birth allows them to have the intended number of children unexpectedly fast. For these families the birth of twins only affects the timing of having children, not family size. Although the desired number of children is unobservable, it is likely to be correlated with the age of the mother. Older mothers would have probably reached their intended number of births compared to younger mothers. So twin births are likely to increase family size for women who experience it later in life, while it would only affect the timing of having children for mothers who experience it early in life \parencite{caceres-delpiano_impacts_2006}. 

In order to address this, as well as the observation that twinning is related to the age of the mother, I run separate regressions for several subsamples based on the age of the mother.  For ages from 21 to 40, I constructed subsamples by taking firstborns in the 2+ sample whose mothers are at least as old as the given age. For instance, 21+ would have mothers who are at least 21 years old, 22+ mothers who are at least 22, and so on, all the way to 40 (40+). I then run both OLS and IV for each of these subsamples for all outcome variables and plotted the coefficient estimates and confidence intervals in \autoref{fig:03}. The general picture reveals that the results remain qualitatively the same when we progressively include older mothers, suggesting that systematic biases due to the age of the mother is not a concern. The details of the Figure are discussed in the results section.

Another challenge to the randomness of twins is the observation that mothers who use fertility enhancing treatments are more likely to have multiple births. Since the use of fertility enhancing treatments is usually unobservable, there is a potential bias from using twin births as an instrument \parencite{braakmann_reconsidering_2016}. Although availability and utilization is low, Assisted Reproductive Technologies (ARTs) such as in vitro fertilizations (IVFs) are becoming more prevalent in South Africa. In fact, South Africa has the second highest number of registered IVF centres in the whole continent in 2019 \parencite{ombelet_ivf_nodate}. Studies analysing ART registry data from sub–Saharan Africa have observed high multiple pregnancy rates associated with the utilization of ARTs \parencite{Botha2018,Dyer2019}. There are indications of a rise in the number of multiple births over time in our sample also that could be associated with the evolution of ARTs. Using the dates of births of all respondents in the 2011 South African Census, I plotted the number of multiple births (i.e., twins, triplets, quadruplets, etc.) per 1000 live births from 1970 onwards. This is shown in \autoref{fig:02}. Although there is no historical data on the utilization of ARTs in South Africa since its introduction in the 1980s (ART registries were established in the late 2000s), the rising trend of multiple births across all population groups is consistent with the introduction and (slow) dissemination of fertility treatments. 

A related threat to the randomness of twin births is the observation that the occurrence of twins is related to the mother's health and health related behaviour. \textcite{bhalotra_twin_2019} show that twinning is strongly related to morbidity, smoking status, availability of reproductive health services and other health indicators even for mothers not using fertility treatments. They also show that the education of the mother is positively associated with twinning, consistent with the hypothesis that educated women are more likely to engage in health seeking behaviour (as well as to seek fertility enhancing treatments). Although the 2011 South African census does not have data on health indicators, I have been able to replicate the conjecture that twinning is more likely in subpopulations that would seek health treatments. \autoref{fig:02} reveals that starting from the mid-1990s the occurrence of multiple births is systematically higher among white South Africans as compared to non-whites. This would make sense as white mothers arguably have both the awareness and the capacity when it comes to utilizing health treatments including ARTs. To check whether this apparent difference poses a challenge, I report the same set of results (first stages, OLS, and 2SLS) using the twins instrument for whites and non-whites. Without going into details (see the results section), the results, reported in \autoref{tab:04}, are robust to division by mother's population group, suggesting that there is no evidence of a potential bias stemming from systematic differences in the occurrence of multiple births.

\textcite{rosenzweig_population_2009} point out a further challenge to the exogeneity of twin births. They argue that the occurrence of twins changes the behaviour of parents in such a way that they reallocate resources away from twins and towards older singleton children. They point out two aspects of twins that affect inter-child allocations: \enquote{(i) the closer spacing of twins, which makes investing in the average quality of the twins more costly compared with investing in non-twins, and (ii) the lower endowments of twins, which will affect the resources allocated to non-twins depending on whether parents reinforce or compensate for endowment differences across children.} \parencite[p. 1152]{rosenzweig_population_2009}. Here the lower endowment of twins refers to the fact that twins usually have lower birth weight, lower survival rate, lower cognitive achievement and so on. If such resource reallocation is taking place, then estimates using twins as an instrument will be biased towards zero.

Following \textcite{angrist_multiple_2010}, I estimated reduced-form twins effects on outcomes in samples in which twins are unlikely to affect family size (i.e., there is no first stage) to see if resource reallocation is a problem. The rationale for focusing on no-first-stage samples is that because the birth of twins has no effect on family size in these subsamples, effects of confounding factors related to twinning should surface. The two subsamples with a zero first stage are those from families with closely spaced births ($ < 2 $ years between the first two births) and those whose mothers were not educated, both from families with at least three children in the 2+ sample.\footnote{These are households who are likely to have large families anyway. In such a case, the birth of twins is likely to have little or no effect on family size.} Columns 1 and 2 in Panel A of \autoref{tab:05} shows that the first stage regressions in such families yield insignificant coefficient estimates at the conventional 5\% level. In Panel B, I estimate reduced form regressions of all four outcomes on the Twin2 dummy. If what \textcite{rosenzweig_population_2009} propose is the case, we ought to see that older singletons have a higher outcome in these no first stage samples. But the opposite is true: in both subsamples, twin births do not significantly affect any of the outcomes for older children. The only exception is the probability of private school attendance in the subsample of uneducated mothers. But even in this case, firstborn children have \textit{lower} not higher likelihood of attending private school, in defiance of the hypothesis that nontwin firstborn children are favoured. Hence, there is no evidence for a reallocation effect that could potentially confound our IV estimates.  

\subsubsection{The Same Sex Instrument}

Instrumenting with the gender mix of siblings in the family was first introduced by \textcite{angrist_children_1998}. They used it as a source of exogenous variation of family size to investigate the labour supply response of mothers to fertility. They explain their rationale for doing so as follows:

\begin{displayquote}[Angrist and Evans (1998, p. 451)]
This instrument exploits the widely observed phenomenon of parental preferences for a mixed sibling-sex composition. In particular, parents of same-sex siblings are significantly and substantially more likely to go on to have an additional child. Because sex mix is virtually randomly assigned, a dummy for whether the sex of the second child matches the sex of the first child provides a plausible instrument for further childbearing among women with at least two children. 
\end{displayquote}

There is plenty of empirical support for the claim that parents prefer to have children of both genders and that this preference drives the decision to have additional children \parencite{norling_measuring_2018,bisbee_local_2015}. The first stage regression on the same sex dummy is also consistent with this result. The results in \autoref{tab:02} show that in both the 2+ and 3+ samples, the same sex dummies have a positive and statistically significant relationship with the number of children born to a mother. The same relationship holds when it is disaggregated by gender.

The results from the same sex IV should also be interpreted as a LATE that applies only to the subpopulation of compliers --- i.e., those families whose number of children was increased as a result of having had two (three) first children of the same sex \parencite{bisbee_local_2015,oberg_casual_2021}. However, just as in the case of the twins instrument, there have been objections to the exclusion restriction with respect to the same sex IV. There is a possibility of gender-specific economies of scale that could reinforce child quality investments when parents have children of the same sex \parencite{clarke_children_2018}. \textcite{rosenzweig_natural_2000} point out that cost efficiencies associated with gender mix will confound the effect of an exogenous increase in the number of children with direct child-rearing cost effects on outcomes. To test this, I run reduced form regressions on subsamples with zero first stages, similar to the no-first-stage analysis for the twins instrument. I consider outcomes for firstborn boys in the 2+ sample, so the appropriate instrument to consider would be the Boy12 dummy. The two subsamples include mothers with less than 2 years of spacing in the first two births and those with no schooling. The results, which are reported in \autoref{tab:05}, show that there is no reduced form relation between the Boy12 instrument and any of the outcomes for the firstborn child. That is, a firstborn male child who has a younger sibling of the same gender does not appear to benefit in anyway as suggested by the same-sex cost advantage hypothesis.\footnote{The only statistically significant effect is observed on the mother's labour force participation in the subsample with close spacing (column 3). But we are focusing here on the direct effects on the child's outcomes. Even though there would be indirect effects, we should have seen it in the regressions where child outcomes are the dependent variables.} Hence, there is no ground to suspect a violation of the exclusion restriction based on household efficiencies related to same sex sibships. 








