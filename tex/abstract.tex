\begin{abstract}
   Children who are raised in large families are usually observed to end up with less education than those from smaller families. A longstanding question in research on fertility is whether this observed negative correlation between sibling size and educational outcomes for children is causal. Using census data from South Africa, this study investigates the effect of family size on the educational attainment of children. Results from OLS regression point to a negative association, as expected. On the other hand, 2SLS estimates using the birth of twins and sibling sex composition as instruments for the number of children show no effect of fertility. Heterogeneity analysis suggests that this null effect holds in virtually all subpopulations with different sociodemographic characteristics. The results are consistent in showing that sibship size has no adverse effects on the educational attainment of children in South Africa.
\end{abstract}